\section{Descripción general del proyecto}
\label{sec:descripcion}

\noindent La simulación de una red de comunicaciones con dispositivos personales requiere modelos que representen fielmente los patrones de movimiento de las personas. De lo contrario, las conclusiones derivadas de dicha simulación pueden ser poco útiles. Para avanzar hacia la definición de un modelo de trayectorias individuales, se propone caracterizar los datos de una base existente que permita modelar trayectorias de forma eficaz.

\vspace{-1ex}
\section{Objetivos y propósitos}
\label{sec:objetivos}

\noindent El objetivo principal del proyecto es obtener una caracterización estadística de las trayectorias individuales. \\ \\
Los propósitos específicos son:
\vspace{-1ex}
\begin{itemize}
    \setlength\itemsep{0pt}
    \item Caracterizar la base de datos para extraer las trayectorias contenidas.
    \item Aplicar un modelo de inteligencia artificial para identificar y analizar dichas trayectorias.
\end{itemize}

\vspace{-1ex}
\section{Alcance del sistema}
\label{sec:alcance}

\noindent El sistema se enfoca en la identificación de trayectorias peatonales individuales y su análisis mediante herramientas de IA. El alcance incluye:
\vspace{-1ex}
\begin{itemize}
    \setlength\itemsep{0pt}
    \item Caracterización de la base de datos existente.
    \item Identificación de trayectorias individuales.
    \item Generación de reportes y visualizaciones de los resultados.
\end{itemize}

\noindent No se incluye la creación de modelos de IA desde cero; se emplearán herramientas y modelos ya existentes.
