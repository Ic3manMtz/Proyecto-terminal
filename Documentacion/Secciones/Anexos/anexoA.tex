\label{anexo:docker}

En la sección \ref{sec:requisitos-sistema} se establece como requisito el uso de Docker y Docker Compose para la ejecución del proyecto. A continuación, se detallan las instrucciones necesarias para su instalación, ya que ambas herramientas son fundamentales para la implementación. Además, se describe el archivo \texttt{docker-compose.yml}, el cual permite crear un contenedor que incluye todas las dependencias requeridas para el correcto funcionamiento del sistema.

\section{¿Qué son Docker y Docker Compose?}
Docker es una plataforma de virtualización ligera que permite desarrollar, empaquetar y ejecutar aplicaciones en contenedores aislados. Un contenedor incluye el código, las dependencias y configuraciones necesarias para que la aplicación se ejecute de manera consistente en cualquier entorno. Esto facilita la portabilidad, escalabilidad y despliegue de software.

Docker Compose es una herramienta que permite definir y ejecutar aplicaciones multicontenedor mediante archivos de configuración YAML. A través de un solo archivo \texttt{docker-compose.yml}, es posible especificar los servicios, redes y volúmenes que componen una aplicación, simplificando así su orquestación.

Estas herramientas son fundamentales en este proyecto para garantizar que el entorno de ejecución sea replicable y controlado, independientemente del sistema operativo o configuración local del usuario.

\section{Instalación en Linux (Ubuntu/Debian)}

Para instalar Docker y Docker Compose en un sistema Linux basado en Debian o Ubuntu, siga los siguientes pasos:

\begin{enumerate}
    \item Actualizar los paquetes del sistema:
    \begin{lstlisting}[
			language=bash,
			caption={Actualizar el sistema.},
			label={cod:update_system}
		]
sudo apt update
sudo apt upgrade
		\end{lstlisting}
    
    \item Instalar Docker:
    \begin{lstlisting}[
			language=bash,
			caption={Instalar Docker.},
			label={cod:install_docker}
		]
sudo apt install docker.io
sudo systemctl enable docker
sudo systemctl start docker
    \end{lstlisting}
    
    \item Verificar que Docker está instalado correctamente:
    \begin{lstlisting}[
			language=bash,
			caption={Verificar instalación de Docker.},
			label={cod:check_docker}
		]
docker --version
    \end{lstlisting}
    
    \item Instalar Docker Compose:
    \begin{lstlisting}[
			language=bash,
			caption={Instalar Docker Compose.},
			label={cod:install_docker_compose}
		]
sudo apt install docker-compose
    \end{lstlisting}
    
    \item Verificar la instalación:
    \begin{lstlisting}[
			language=bash,
			caption={Verificar instalación de Docker Compose.},
			label={cod:check_docker_compose}
		]
docker-compose --version
    \end{lstlisting}
\end{enumerate}

\section{Instalación en Windows}

Para instalar Docker y Docker Compose en Windows, se recomienda utilizar Docker Desktop, que incluye ambas herramientas de forma integrada.

\begin{enumerate}
    \item Acceder al sitio oficial: \href{https://www.docker.com/products/docker-desktop/}{https://www.docker.com/products/docker-desktop/}
    
    \item Descargar el instalador correspondiente para Windows.

    \item Ejecutar el instalador y seguir el asistente de instalación.

    \item Reiniciar el sistema si es necesario.

    \item Verificar que Docker y Docker Compose estén correctamente instalados desde la terminal de Windows (PowerShell o CMD):
    \begin{lstlisting}[
			language=bash,
			caption={Iniciar contenedor del proyecto.},
			label={cod:start_container}
		]
docker --version
docker-compose --version
    \end{lstlisting}
\end{enumerate}

\textbf{Nota:} Docker Desktop requiere que la virtualización esté habilitada en la BIOS del sistema. También es necesario contar con Windows 10 o superior.

\section{Descripción del archivo \texttt{docker-compose.yml}}
El archivo \texttt{docker-compose.yml} permite definir y configurar el entorno de ejecución del proyecto utilizando un contenedor de Docker. A continuación, se presenta su contenido y una explicación de cada uno de sus elementos:

\vspace{3mm}
\begin{lstlisting}[
			language=bash,
			caption={Archivo docker-compose.yml},
			label={cod:docker_compose_file}
		]
version: "3.8"

services:
  data-analysis:
    image: python:3.13-bookworm
    container_name: data-analysis
    runtime: nvidia
    tty: true
    stdin_open: true
    volumes:
      - ./:/app
      - python-packages:/usr/local/lib/python3.13/site-packages
    command: sh -c "pip install -r requirements.txt && pip install -e . && python3 /app/src/main.py"
    working_dir: /app
    environment:
      - PYTHONPATH=/app

volumes:
  python-packages:
\end{lstlisting}

A continuación se explica el propósito de cada sección:

\begin{itemize}
  \item \texttt{version: "3.8"}\\
  Define la versión del esquema de Docker Compose utilizado. La versión 3.8 es compatible con la mayoría de las características modernas de Docker.

  \item \texttt{services} → \texttt{data-analysis}\\
  Se define un servicio llamado \texttt{data-analysis}, que representa el contenedor principal del proyecto.

  \item \texttt{image: python:3.13-bookworm}\\
  Utiliza una imagen oficial de Python 3.13 basada en Debian Bookworm como entorno base.

  \item \texttt{container\_name: data-analysis}\\
  Asigna un nombre personalizado al contenedor para facilitar su identificación.

  \item \texttt{runtime: nvidia}\\
  Indica que el contenedor utilizará el runtime de NVIDIA para permitir acceso a la GPU. Requiere tener instalado \texttt{nvidia-docker}.

  \item \texttt{tty: true} y \texttt{stdin\_open: true}\\
  Habilitan la interacción con el terminal del contenedor, lo que es útil para ejecutar comandos manuales si es necesario.

  \item \texttt{volumes}\\
  \begin{itemize}
    \item \texttt{./:/app}: Monta el directorio actual del proyecto como \texttt{/app} dentro del contenedor.
    \item \texttt{python-packages:/usr/local/lib/python3.13/site-packages}: Crea un volumen persistente para las bibliotecas de Python instaladas.
  \end{itemize}

  \item \texttt{command}\\
  Ejecuta una serie de comandos cuando el contenedor inicia: instala las dependencias del archivo \texttt{requirements.txt}, instala el proyecto en modo editable (\texttt{pip install -e .}) y ejecuta el archivo \texttt{main.py}.

  \item \texttt{working\_dir: /app}\\
  Establece el directorio de trabajo dentro del contenedor como \texttt{/app}.

  \item \texttt{environment}\\
  Define la variable de entorno \texttt{PYTHONPATH} para que Python pueda encontrar correctamente los módulos dentro del proyecto.
  
  \item \texttt{volumes → python-packages}\\
  Declara un volumen persistente llamado \texttt{python-packages}, que se utiliza para almacenar los paquetes instalados sin perderlos entre reinicios del contenedor.
\end{itemize}

Este archivo permite que el entorno de desarrollo sea fácilmente replicable y ejecutable, sin necesidad de instalar manualmente dependencias o configurar rutas en el sistema anfitrión.
