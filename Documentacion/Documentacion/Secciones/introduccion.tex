	\section{Introducci\'on}
	\subsection{Descripci\'on general del proyecto}
	\noindent La simulaci\'on de una red de comunicaciones en donde intervienen dispositivos personales de comunucaci\'on requiere contar con modelos que representen fielmente los patrones de movimiento de las personas. De otra manera, la utilidad de las conclusiones que se puedan obtener de esa simulaci\'on es limitada. 

	\noindent Para avanzar hacia la definici\'on de un modelo de trayectorias individuales, se propone la caracterización de los datos de una base de datos para poder modelar trayectorias eficazmente. 
	
	\subsection{Objetivos y prop\'ositos}
	\noindent El objetivo principal del proyecto es contar con una caracterizaci\'on de las trayectorios. Identificando las caracter\'isticas estad\'isticas que las componen.

	\noindent Los prop\'ositos del proyecto son:
	\begin{itemize}
		\item Caracterizar la base de datos para obtener las trayectorias contenidas.
		\item Usar un modelo de IA que permita identificar y caracterizar las trayectorias obtenitas.
	\end{itemize}

	\subsection{Alcance del sistema}
	\noindent El sistema se enfoca en la identificaciión de trayectorias peatonales individuales y su an\'alisis utilizando herramientas de IA. El alcance incluye:
	\begin{itemize}
		\item Caracterizar la base existente.
		\item Identificación de trayectorias peatonales individuales.
		\item Generaci\'on de reportes y visualizaciones de los resultados obtenidos.
	\end{itemize}
	\noindent No se considera dentro del alcance la implementaci\'on de modelos de IA desde cero; se utilizar\'an herramientas y modelos existentes.

	\newpage